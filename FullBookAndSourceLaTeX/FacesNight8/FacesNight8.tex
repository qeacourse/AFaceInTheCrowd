\chapter{Night 8}

\section{Eigenfaces Paper}

Check out \href{https://drive.google.com/open?id=0B7LNBbaxYFujTkwxY1BZZEE2eWM}{\textit{Eigenfaces for Recognition}}, an early paper on eigenfaces, by M. Turk and A. Pentland. You have most of the tools to understand this paper, but the writing style might be unfamiliar (intense!). Spend 1 hour on this paper and then feel free to move on. The first 6 pages of this paper describe the use of eigenfaces in face recognition.
Check out other sources as well. Wikipedia is pretty useful for eigenfaces, and \href{https://drive.google.com/open?id=0B7LNBbaxYFujLUFfNThyT2JqNmc}{this} later paper talks about eigenfaces and an extension called Fisherfaces (not fish faces).

\begin{prob}
We are asking you read this paper for several reasons. We hope that it highlights and synthesizes all the material you've learned in this module. It will also give you practice reading a technical paper, which is a skill you'll continue to develop over your career.
\begin{enumerate}
    \item In what ways was your approach to implementing the eigenfaces algorithm similar or different from the authors' approach?
    \item In what ways did your understanding of the eigenfaces algorithm change after reading the paper?
    \item Were there places in the reading that you ``got stuck?'' If so, how did you address that?
    \item What questions do you have after reading the paper?
\end{enumerate}
\end{prob}

\section{Beginning the Project}

In this project you will extend the work you have already done on facial recognition and feature detection by analyzing the performance of an existing algorithm within a real context. We know that facial recognition and other forms of feature detection algorithms are incredibly powerful, but they are often prone to failure, and those failures can have very real consequences on people's lives. This is a new formulation of this project where we are challenging you to think deeply about the contexts and consequences for facial/feature recognition algorithms. To prepare for tomorrow's in-class ideation activities, we ask you to do two things:
\begin{enumerate}
    \item Read the project description, which can be found in the next chapter (Chapter~\ref{ch:facesproj}), and write down any questions you find yourself asking.
    \item Fill out \href{https://docs.google.com/forms/d/e/1FAIpQLSfMbSQudUCL1xaCBpK-fzNweDwRYuSitGRDiA5PbsVmnrnPbA/viewform?usp=sf_link}{this partner survey} by midnight tonight, Feb. 19th. We will announce the teams tomorrow morning.
\end{enumerate}

\pagebreak
\shipoutAnswer