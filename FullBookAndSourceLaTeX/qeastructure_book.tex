
%%%%%%%%%%%%%%%%%%%%%%%%%%%%%%%%%%%%%%%%%
% eBook
% Structural Definitions File
% Version 1.0 (29/12/14)
%
% Created by:
% Vel (vel@latextemplates.com)
% 
% This file has been downloaded from:
% http://www.LaTeXTemplates.com
%
% License:
% CC BY-NC-SA 3.0 (http://creativecommons.org/licenses/by-nc-sa/3.0/)
%
%%%%%%%%%%%%%%%%%%%%%%%%%%%%%%%%%%%%%%%%%

%----------------------------------------
%	REQUIRED PACKAGES
%----------------------------------------
\usepackage{titlesec}
\usepackage{enumitem}
%\usepackage{optidef}
\usepackage{multicol}
\usepackage{graphicx}
\usepackage{import}
\usepackage{amsmath,bm}
\usepackage{siunitx}
\usepackage{verbatim, ifthen}
\usepackage[skins,breakable]{tcolorbox}
\usepackage[utf8]{inputenc} % Required for inputting international characters
\usepackage[T1]{fontenc} % Output font encoding for international characters
\usepackage[osf]{libertine} % Use the Libertine font
\usepackage{microtype} % Improves character and word spacing
\usepackage{tikz} % Required for drawing custom shapes
\usepackage{wallpaper} % Required for setting background images (title page)
\usepackage[autolinebreaks,useliterate]{mcode}
\usepackage{colortbl}
\usepackage{longtable}
\usepackage{pdfpages}

\usepackage[unicode=true,bookmarks=true,bookmarksnumbered=false,bookmarksopen=false,breaklinks=false,pdfborder={0 0 1},backref=section,colorlinks=true]{hyperref} % PDF meta-information specification
\hypersetup{
    colorlinks=true,
    linkcolor=blue,
    filecolor=magenta,
    urlcolor=green,
}
\usepackage{caption}
\usepackage{hypcap}
\usepackage{wrapfig} %allows wrapping text around a figure


%----------------------------------------
%	PAPER, MARGIN AND HEADER/FOOTER SIZES
%----------------------------------------

\setstocksize{11in}{8.5in}
\setlrmarginsandblock{0.15\paperwidth}{*}{1} % Left and right margin
\setulmarginsandblock{0.2\paperwidth}{*}{1}  % Upper and lower margin

%----------------------------------------
%	FOOTNOTE CUSTOMIZATION
%----------------------------------------

\renewcommand{\foottextfont}{\itshape\footnotesize} % Font settings for footnotes
\setlength{\footmarkwidth}{-.1em} % Space between the footnote number and the text
\setlength{\footmarksep}{.1em} % Space between multiple footnotes on the same page
\renewcommand*{\footnoterule}{} % Remove the rule above the first footnote
\setlength{\skip\footins}{1\onelineskip} % Space between the body text and the footnote

%----------------------------------------
%	HEADER AND FOOTER FORMATS
%----------------------------------------

\makepagestyle{mio} % Define a new custom page style
\setlength{\headwidth}{\textwidth} % Header the same width as the text
\makeheadrule{mio}{\textwidth}{0.1mm} % Header rule height
\makeoddhead{mio}{\scriptsize{\theauthor\hskip.2cm\vrule\hskip.2cm\itshape{\thetitle}}}{}{} % Header specification
\makeoddfoot{mio}{}{\scriptsize {\thepage \quad \vrule \quad \thelastpage}}{} % Footer specification
\makeoddfoot{plain}{}{\footnotesize {\thepage \quad \vrule \quad \thelastpage}}{} % Pages of chapters
\pagestyle{mio} % Set the page style to the custom style defined above

%---------------------------------------
%	PART FORMAT
%---------------------------------------
\renewcommand{\partname}{Module}
\renewcommand{\partnamefont}{\centering\sffamily\itshape\Huge} % Part name font specification
\renewcommand{\partnumfont}{\sffamily\Huge} % Part number font specification
\renewcommand{\parttitlefont}{\centering\sffamily\scshape} % Part title font specification
\renewcommand{\beforepartskip}{\null\vskip.118\textheight} % Whitespace above the part heading

%--------------------------------------
%	CHAPTER FORMAT
%--------------------------------------

\makechapterstyle{Tufte}{ % Define a new chapter style
\renewcommand{\chapterheadstart}{\vskip1.0 \onelineskip} % Whitespace before the chapter starts
\renewcommand{\printchaptername}{\large\itshape\chaptername} % "Chapter" text font specification
\renewcommand{\printchapternum}{\LARGE\thechapter \\} % Chapter number font specification
\renewcommand{\afterchapternum}{} % Space between the chapter number and text
\renewcommand{\printchaptertitle}[1]{ % Chapter title font specification
\raggedright
\itshape\LARGE{##1}}
\renewcommand{\afterchaptertitle}{\vskip1.0 \onelineskip}
}
\chapterstyle{Tufte} % Set the chapter style to the custom style defined above

%--------------------------------------
%	SECTION FORMAT
%--------------------------------------

\setsecheadstyle{\sethangfrom{\noindent ##1}\raggedright\sffamily\itshape\Large} % Section title font specification
\setbeforesecskip{-.6\onelineskip} % Whitespace before the section
\setaftersecskip{.3\onelineskip} % Whitespace after the section
%\newcommand{\sectionbreak}{\clearpage}
%--------------------------------------
%	SUBSECTION FORMAT
%--------------------------------------

\setsubsecheadstyle{\sethangfrom{\noindent  ##1}\raggedright\sffamily\large\itshape} % Subsection title font specification
\setbeforesubsecskip{-.5\onelineskip} % Whitespace before the subsection
\setaftersubsecskip{.2\onelineskip} % Whitespace after the subsection
%\newcommand{\subsectionbreak}{\clearpage}
%--------------------------------------
%	SUBSUBSECTION FORMAT
%--------------------------------------

\setsubsubsecheadstyle{\sethangfrom{\noindent ##1}\raggedright\sffamily\itshape} % Subsubsection title font specification
\setbeforesubsubsecskip{-.5\onelineskip} % Whitespace before the subsubsection
\setaftersubsubsecskip{.1\onelineskip} % Whitespace after the subsubsection

%--------------------------------------
%	CAPTION FORMAT
%--------------------------------------

\captiontitlefont{\itshape\footnotesize} % Caption font specification
\captionnamefont{\footnotesize} % "Caption" text font specification

%--------------------------------------
%	QUOTATION ENVIRONMENT FORMAT
%--------------------------------------

\renewenvironment{quotation}
{\par\leftskip=1em\vskip.5\onelineskip\em}
{\par\vskip.5\onelineskip}

%--------------------------------------
%	QUOTE ENVIRONMENT FORMAT
%--------------------------------------

\renewenvironment{quote}
{\list{}{\em\leftmargin=1em}\item[]}{\endlist\relax}

%--------------------------------------
%	EXERCISES AND SOLUTIONS ENVIRONMENT FORMAT
%--------------------------------------


\numberwithin{Exercise}{chapter} % include the chapter number in the exercise
\numberwithin{Answer}{chapter} % include the chapter number in the answer


\renewcommand{\ExerciseName}{Exercise} % Use Exercise as the name
\renewcommand{\AnswerName}{Solution} % Use Solution as the name

\renewcommand{\ExerciseHeader}{\centerline{ \large\textbf{\ExerciseName~\ExerciseHeaderNB} \smallskip}} % Control how it looks

\renewcommand{\AnswerHeader}{\centerline{ \large\textbf{\AnswerName~\ExerciseHeaderNB}
\smallskip}} % Control how it looks

% the Solution environment
\newenvironment{sol}
        {\ifthenelse{\isundefined{\showsolutions}}
                    {\expandafter\comment}
                   {\begin{tcolorbox}[enhanced jigsaw, breakable]\begin{Answer}}
                    }
         {\ifthenelse{\isundefined{\showsolutions}}
                    {\expandafter\endcomment}
                    {\end{Answer}\end{tcolorbox}}
          }

% the Exercise environment
\newenvironment{prob}
        {\ifthenelse{\isundefined{\showexercises}}
                    {\newpage\expandafter\comment}
                    {\begin{tcolorbox}[enhanced jigsaw, breakable]\begin{Exercise}}
                    }
         {\ifthenelse{\isundefined{\showexercises}}
                    {\expandafter\endcomment}
                    {\end{Exercise}\end{tcolorbox}}
          }
          
% instructor notes environment
\newenvironment{ournotes}
        {\ifthenelse{\isundefined{\shownotes}}
                    {\expandafter\comment} {\begin{tcolorbox}[enhanced,borderline ={4pt}{-2pt}{red!90!black}, colframe=red!95!black]\textbf{Instructors' Notes:}}
                    }
         {\ifthenelse{\isundefined{\shownotes}}
                    {\expandafter\endcomment}
                    {\end{tcolorbox}}
          }

%--------------------------------------
%	MISCELLANEOUS DOCUMENT SPECIFICATIONS
%--------------------------------------

\setlength{\parindent}{1em} % Paragraph indentation

\midsloppy % Fewer overfull lines - used in the memoir class and allows a setting somewhere between \fussy and \sloppy

\checkandfixthelayout % Tell memoir to implement the above

\newcommand{\bi}{\begin{itemize}}
\newcommand{\ei}{\end{itemize}}
\newcommand{\be}{\begin{enumerate}}
\newcommand{\ee}{\end{enumerate}}
\newcommand{\beq}{\begin{equation}}
\newcommand{\eeq}{\end{equation}}

\newcommand{\twobytwo}[4]{\ensuremath{\begin{bmatrix} #1 & #2 \\ #3 & #4\end{bmatrix}}}
\newcommand{\threebythree}[9]{\ensuremath{\begin{bmatrix} #1 & #2 & #3 \\ #4 & #5 & #6 \\ #7 & #8 & #9 \end{bmatrix}}}
\newcommand{\threebytwo}[6]{\ensuremath{\begin{bmatrix} #1 & #2  \\ #3 & #4  \\ #5 & #6  \end{bmatrix}}}
\newcommand{\twobythree}[6]{\ensuremath{\begin{bmatrix} #1 & #2 & #3 \\ #4 & #5 & #6  \end{bmatrix}}}
\newcommand{\onebytwo}[2]{\ensuremath{\begin{bmatrix} #1 & #2 \end{bmatrix}}}
\newcommand{\onebythree}[3]{\ensuremath{\begin{bmatrix} #1 & #2 & #3 \end{bmatrix}}}

\newcommand{\twobyone}[2]{\ensuremath{\begin{bmatrix} #1 \\ #2 \end{bmatrix}}}
\newcommand{\threebyone}[3]{\ensuremath{\begin{bmatrix} #1 \\ #2 \\ #3 \end{bmatrix}}}

\newcommand{\fourbyone}[4]{\ensuremath{\begin{bmatrix} #1 \\ #2 \\ #3 \\ #4 \end{bmatrix}}}

\newcommand{\fourbytwo}[8]{\ensuremath{\begin{bmatrix} #1 & #2 \\ #3 & #4 \\ #5 & #6 \\ #7 & #8 \end{bmatrix}}}

\newcommand{\That}{\hat{\mathbf{T}}}
\newcommand{\Nhat}{\hat{\mathbf{N}}}
\newcommand{\Bhat}{\hat{\mathbf{B}}}

\newcommand{\T}{\mathbf{T}}
\newcommand{\R}{\mathbf{R}}
\newcommand{\A}{\mathbf{A}}
\newcommand{\B}{\mathbf{B}}
\newcommand{\F}{\mathbf{F}}
\renewcommand{\U}{\mathbf{U}}
\newcommand{\I}{\mathbf{I}}
\newcommand{\V}{\mathbf{V}}
\renewcommand{\L}{\mathbf{L}}
\newcommand{\M}{\mathbf{M}}
\renewcommand{\C}{\mathbf{C}}
\renewcommand{\S}{\mathbf{S}}

\renewcommand{\v}{\mathbf{v}}
\renewcommand{\t}{\mathbf{t}}
\renewcommand{\u}{\mathbf{u}}
\renewcommand{\r}{\mathbf{r}}
\renewcommand{\a}{\mathbf{a}}
\newcommand{\e}{\mathbf{e}}
\newcommand{\x}{\mathbf{x}}
\newcommand{\y}{\mathbf{y}}
\newcommand{\z}{\mathbf{z}}
\renewcommand{\b}{\mathbf{b}}
\newcommand{\f}{\mathbf{f}}

\newcommand{\ihat}{\hat{\textbf{\i}}}
\newcommand{\jhat}{\hat{\textbf{\j}}}
\newcommand{\khat}{\hat{\mathbf{k}}}
\newcommand{\Om}{\bm{\omega}}
\newcommand{\Lagr}{\mathcal{L}}
