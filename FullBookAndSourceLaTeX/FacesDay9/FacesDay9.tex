\chapter{Day 9}

\section*{Schedule}
\begin{itemize}
\item 0900--0930		Debrief Eigenfaces Paper
\item 0930--1030        Project Ideation
\item 1030--1045		Coffee Break
\item 1045--1225        Project Worktime
\end{itemize}

\section{Debrief Eigenfaces Paper}

One of the goals of QEA is to develop your abilty to read technical papers and implement the main ideas. Much of engineering practice is based on building on the incredible work of those who came before us, and being able to critically read technical writing is a skill to be developed. Similarly, identifying the particularly helpful, or challenging, portions of a paper can help you improve your own technical writing. Please consider the following prompts:
\begin{prob}
\begin{enumerate}
\item What did the \textbf{authors set out to show}?  Why did the authors choose this particular goal?
\item What were the \textbf{best parts} of the paper? Was the approach to facial recognition easy to follow? What did the authors do particularly well?
\item Identify the \textbf{missing pieces} in the paper.  These could be missing analyses, missing punchline graphs, missing equations, etc. Another way to think about this is which of the main conclusions of the paper are you still the most skeptical of.  Your skepticism could arise from a lack of evidence, or because the evidence was not presented in a clear and easily digestible format.
\end{enumerate}
\end{prob}

\section{Project Ideation}
You should be seated with your partner for the rest of class.

\subsection{User Ideation Extravaganza}

There are A LOT of possible questions you could propose for this project. In the project document we prompt you to \textit{choose an important question related to feature recognition, detection, or classification}. This should be rooted in a real context, and you will likely not be able to answer it entirely. Break off a small subquestion that you think you can answer in one week through an analytical approach that utilizes eigenfaces, another facial recognition algorithm, or linear regression. We recognize that this is a very open ended and somewhat ambiguous prompt, but we feel you are up to the challenge! We will be taking the remainder of class to generate lots of possible questions, refine ideas, and try to generate teams around shared interests.

\begin{prob}
\begin{enumerate}
\item Do this part individually [10 minutes]: Write down as many different ideas for possible questions related to facial/feature recognition/detection/classification as you can on different sticky notes. Go wild!
\item Do this part with your table and the neighboring table [10 minutes]: Our goal now is to get all of the questions up on the back board and to group them (in other words, all of the questions around flagging and the potential bias in that process could go into one group). Draw on the board as needed. This may feel a bit chaotic but we will help you get there.
\item Do this part with your partner [5 minutes]: Walk around the room with your partner, reading the idea clusters from the other table groups. Get inspired! 
\end{enumerate}
\end{prob}


\subsection{Pair Project Ideation}
This next stage is going to give you an opportunity to work with your partner to develop a complete project idea and to "pitch" it to the class. 

\begin{prob}
\begin{enumerate}
\item With a partner at your table, select a question (or group of questions) from the boards. Grab the appropriate sticky notes and bring them to your table. [5 minutes]
\item Collaboratively develop an idea that identifies a question, it's context, and how you could perform some analysis. Do some internet research to find out more about your question and its context. What are the ethical issues associated with your topic? What is known and what is still in question? Fill out the \href{https://drive.google.com/file/d/1JhAOYK4TXdeA8rMV2zZwerRZf25l1zku/view?usp=sharing}{project pitch handout}. You will need to define the question itself, the real world context, and details about the critical concepts from this module that will allow you to perform the desired analysis. [20 minutes]
\end{enumerate}
\end{prob}

\section{Project Worktime}
Get started on the details of your project. You should consider this an extension of the project ideation time. Play with ideas, and hopefully, by the end of class you'll feel like you've settled on an idea and have a direction to go in. Use this time to chat with the faculty.