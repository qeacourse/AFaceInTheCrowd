%LV begin insert from D9
\chapter{Day 11: Consequences of facial recognition and project work time}

\section{Schedule}
\begin{itemize}
\item 09:00 - 09:30			Video and discussion
\item 09:30 - 10:00		Considering consequences
\item 10:00 - 10:15		Break
\item 10:15 - 12:30		Project work
\end{itemize}

Sit with your project partner and another pair. You will not need to turn anything in today.

\section{Video and discussion [30 min]}

While watching Algorithmic Justice League founder \href{https://www.ted.com/talks/joy_buolamwini_how_i_m_fighting_bias_in_algorithms#t-459194}{Joy Buolamwini's TED talk} on algortihmic bias, think about the following questions:%and/or https://www.nytimes.com/2018/06/21/opinion/facial-analysis-technology-bias.html and/or https://enterprisersproject.com/article/2019/1/ai-bias-9-questions-leaders-should-ask
\be
\item What causes bias in AI?
\item What are the potential consequences of bias in AI?
\item How can we systematically prevent bias in the technology we create?
\ee

Discuss the video with your table. How do you feel after watching it? What stood out to you? You may, if you wish, use the following questions to guide your discussion.
\be

\item Joy talks about bias in AI creating "exclusionary experiences and discriminatory practices." What examples of each of these problems did Joy give? What other negative results of bias in AI can you imagine?
\item Think about Joy's story. What can we learn from they way she acted after noticing a problem with AI?
\item How will you prevent bias in the technology you create?
\item How can engineers systematically predict and prevent negative impacts of new technology in general?
\ee

\section{Considering consequences [30 min]}
Now you will think about preventing negative consequences of technology in the context of your projects. For this activity, you will briefly switch projects with the other pair at your table to think about the consequences associated with their project.
\be
\item [\textbf{10 min}] Each pair should explain their project idea to the rest of the table. The other pair should then ask questions to try to understand the  project in enough detail to think about potential consequences.
\item [\textbf{10 min}] Each pair should then consider the consequences of the technology the other pair is analyzing or creating. Consider whom could be helped by this technology, but focus on the question:  \textbf{whom could be hurt, and how?} Take notes that you can share with the other team.
\item [\textbf{10 min}] Each pair should report their findings to the other pair. Discuss as a table. Each pair should then turn back to their own project and make a plan for how they will address the potential for harm by their project or project topic.
\ee

\section{Break [15 mins]}

Which dining hall beverages have the lowest carbon and water footprints? Hmm...

\section{Project work [1.75 hr]}

As you work on your projects, make sure to address the consequences you identified earlier in this class. Addressing consequences could involve analysis, background research, or improving your algorithm or training data.
%LV end insert from D9